%-----------------------------------------------------
% index key words
%-----------------------------------------------------
\index{matrix!multiplication}

%-----------------------------------------------------
% name, leave blank
% title, if the exercise has a name i.e. Hilbert's matrix
% difficulty = n, where n is the number of stars
% origin = "\cite{ref}"
%-----------------------------------------------------
\begin{Exercise}[
name={},
title={}, 
difficulty=0,
origin={\cite{JH}}]
    Show that if \( G \) has a row of zeros then \( GH \)
    (if defined) has a row of zeros.
    Does the same statement hold for columns?
\end{Exercise}

\begin{Answer}
      The \( i \)-th row of \( GH \) is made up of the products of
      the \( i \)-th row of \( G \) with the columns of \( H \).
      The product of a zero row with a column is zero.

      It works for columns if stated correctly:~if \( H \) has a column of
      zeros then \( GH \) (if defined) has a column of zeros.
\end{Answer}
