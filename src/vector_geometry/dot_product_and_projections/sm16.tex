%-----------------------------------------------------
% index key words
%-----------------------------------------------------
\index{dot product}

%-----------------------------------------------------
% name, leave blank
% title, if the exercise has a name i.e. Hilbert's matrix
% difficulty = n, where n is the number of stars
% origin = "\cite{ref}"
%-----------------------------------------------------
\begin{Exercise}[
name={},
title={}, 
difficulty=0,
origin={\cite{SM}}]
Consider the following statement: ``If $\vec{a}\dotprod\vec{b}=\vec{a}\dotprod\vec{c}$ then $\vec{b}=\vec{c}$.''
\Question If the statement is true, prove it. If the statement is false, provide a counterexample.
\Question If we specify $\vec{a}\neq\vec{0}$, does that change the result?
\end{Exercise}

\begin{Answer}
\Question The statement is false: Let $\vec{a}$ be any non-zero vector, let $\vec{b}$ be any non-zero vector that is orthogonal to $\vec{a}$, and let $\vec{c} = -\vec{b}$. Then the antecedent of the statement is true, since both sides are equal to 0, while the consequent is false.
\Question No.
\end{Answer}
