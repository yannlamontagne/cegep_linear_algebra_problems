%-----------------------------------------------------
% index key words
%-----------------------------------------------------
\index{dot product}
\index{algebraic properties}

%-----------------------------------------------------
% name, leave blank
% title, if the exercise has a name i.e. Hilbert's matrix
% difficulty = n, where n is the number of stars
% origin = "\cite{ref}"
%-----------------------------------------------------
\begin{Exercise}[
name={},
title={}, 
difficulty=0,
origin={\cite{JH}}]
    Suppose that \( \vec{u}\dotprod\vec{v}=\vec{u}\dotprod\vec{w} \)
    and \( \vec{u}\neq\vec{0} \).
    Must \( \vec{v}=\vec{w} \)?
\end{Exercise}

\begin{Answer}
      No.
      These give an example.
      \begin{equation*}
        \vec{u}=\colvec[r]{1 \\ 0}
        \quad
        \vec{v}=\colvec[r]{1 \\ 0}
        \quad
        \vec{w}=\colvec[r]{1 \\ 1}
      \end{equation*}  
\end{Answer}
