%-----------------------------------------------------
% index key words
%-----------------------------------------------------
\index{lines}
\index{lines!skew}
\index{planes}

%-----------------------------------------------------
% name, leave blank
% title, if the exercise has a name i.e. Hilbert's matrix
% difficulty = n, where n is the number of stars
% origin = "\cite{ref}"
%-----------------------------------------------------
\begin{Exercise}[
name={},
title={}, 
difficulty=0,
origin={\cite{MB}}]
Given the following: parametric equations of two skew lines $L_1$ and $L_2$, two nonparallel planes $P_1$ and $P_2$, coordinates of the point $Q$, that does not lie on $L_1$, $L_2$, $P_1$, nor $P_2$. 

\Question Does a line that is perpendicular to $L_1$ and passes through $Q$ exist?
If so, is it unique? How would the equation of such a line (if it exist), be obtained with the provided information?

\Question Does a line parallel to the intersection of the planes $P_1$ and $P_2$ and passes through $Q$ exist? If so, is it unique?
How would the equation such a line (if it exist), be obtained with the provided information?

\Question Does a plane containing $L_1$ and $L_2$ exists? Is so, is it unique? How would the equation of such a plane (if it exists), be obtained with the provided information?
\end{Exercise}

\begin{Answer}
\Question Such a line exists, but it is not unique.  One of such line is obtained by finding a vector $\vec{d}$ perpendicular to the direction vector of $L_1$, then the equation of the line is given by $\vec{x}=Q+t\vec{d}$ where $t\in\Re$.

\Question Such a line exists and is unique.  The line can be obtained by finding a vector $\vec{d}$ parallel to the intersection of $L_1$ and $L_2$, then the equation of the line is given by $\vec{x}=Q+t\vec{d}$ where $
t\in\Re$.

\Question Such a plane does not exist since the lines are skew.
\end{Answer}
