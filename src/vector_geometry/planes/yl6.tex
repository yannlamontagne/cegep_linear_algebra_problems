%-----------------------------------------------------
% index key words
%-----------------------------------------------------
\index{plane}
\index{intersection, intersetion of two planes, intersection!planes}

%-----------------------------------------------------
% name, leave blank
% title, if the exercise has a name i.e. Hilbert's matrix
% difficulty = n, where n is the number of stars
% origin = "\cite{ref}"
%-----------------------------------------------------
\begin{Exercise}[
name={},
title={}, 
difficulty=0,
origin={\cite{YL}}]
Given two planes:
\[
\begin{array}{llrrrrrrrr}
\mathcal{P}_1 & : & \; x +z & = 1\\
\mathcal{P}_2 & : & \; y+z & =1
\end{array}
\]
\Question Give a point of intersection of the two planes, by inspection.
\Question Give a geometrical argument to explain why the intersection of the two planes is a line.
\Question Find the direction vector for the intersection of the two planes without solving for the solution set.  Justify.
\Question Find the solution set of the system of linear equations determined by $\mathcal{P}_1$ and $\mathcal{P}_2$ by only using part a) and part c).
\end{Exercise}

\begin{Answer}
\Question $(0,0,1)$
\Question The normal of both planes are not parallel, hence their inclination are not the same therefore their intersection is a line.
\Question $\vec{d}=\vec{n}_1\times\vec{n}_2=(-1,-1,1)$
\Question $(x,y,z)=(0,0,1)+t(-1,-1,1)$ where $t\in\Re$.
\end{Answer}
