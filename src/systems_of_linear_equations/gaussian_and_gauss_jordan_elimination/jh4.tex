%-----------------------------------------------------
% index key words
%-----------------------------------------------------
\index{quadratic equation, parabola}


%-----------------------------------------------------
% name, leave blank
% title, if the exercise has a name i.e. Hilbert's matrix
% difficulty = n, where n is the number of stars
% origin = "by name \cite{ref}"
%-----------------------------------------------------
\begin{Exercise}[
name={},
title={}, 
difficulty=0,
origin={\cite{JH}}]
Find the coefficients
\( a \), \( b \), and \( c \) so that the graph of \( f(x)=ax^2+bx+c \) 
passes through the points \( (1,2) \), \( (-1,6) \), and \( (2,3) \).
\end{Exercise}

\begin{Answer}
Because \( f(1)=2 \), \( f(-1)=6 \), and \( f(2)=3 \) we get
a linear system.
\begin{equation*}
\begin{linsys}{3}
1a  &+  &1b  &+  &c  &=  &2  \\
1a  &-  &1b  &+  &c  &=  &6  \\
4a  &+  &2b  &+  &c  &=  &3  
\end{linsys}
\end{equation*}
After performing Gaussian elimination we obtain
\begin{equation*}
\begin{linsys}{3}
a  &+  &b  &+  &c  &=  &2  \\
&   &-2b&   &   &=  &4  \\
&   &   &   &-3c&=  &-9 
\end{linsys}
\end{equation*}
which shows that the solution is \( f(x)=1x^2-2x+3 \).  

\end{Answer}
