%-----------------------------------------------------
% index key words
%-----------------------------------------------------
\index{solution set}

%-----------------------------------------------------
% name, leave blank
% title, if the exercise has a name i.e. Hilbert's matrix
% difficulty = n, where n is the number of stars
% origin = "by name \cite{ref}"
%-----------------------------------------------------
\begin{Exercise}[
name={},
title={}, 
difficulty=0,
origin={\cite{KK}}]
If a system of equations has more equations than variables, can it
have a solution? If so, give an example and if not, explain why.
\end{Exercise}

\begin{Answer}
For example, $x+y=1,2x+2y=2,3x+3y=3$ has an infinitely many solutions.
\end{Answer}
