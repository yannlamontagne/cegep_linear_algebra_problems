%-----------------------------------------------------
% index key words
%-----------------------------------------------------
\index{system of linear equations}
\index{particular solution}
\index{constants}


%-----------------------------------------------------
% name, leave blank
% title, if the exercise has a name i.e. Hilbert's matrix
% difficulty = n, where n is the number of stars
% origin = "by name \cite{ref}"
%-----------------------------------------------------
\begin{Exercise}[
name={},
title={}, 
difficulty=0,
origin={\cite{YL}}]
Given the linear system
\[\systeme{x-y+z=b_1,2x-2y-2z=b_2,x +3y-5z=b_3}.\]  Determine the $b_i$ if the linear system has the particular solution $(3,-2, 1)$.
\end{Exercise}

\begin{Answer}
$b_1=6,\;b_2=8,\;b_3=-8$
\end{Answer}
