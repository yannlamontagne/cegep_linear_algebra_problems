%-----------------------------------------------------
% index key words
%-----------------------------------------------------
\index{basis}
\index{linear combination}

%-----------------------------------------------------
% name, leave blank
% title, if the exercise has a name i.e. Hilbert's matrix
% difficulty = n, where n is the number of stars
% origin = "\cite{ref}"
%-----------------------------------------------------
\begin{Exercise}[
name={},
title={}, 
difficulty=0,
origin={\cite{JH}}]
If a subset is not a basis, can linear combinations be not unique?
If so, must they be?
\end{Exercise}

\begin{Answer}
      Here is a subset of $\Re^2$ that is not a basis, and two different
      linear combinations of its elements that sum to the same vector.
      \begin{equation*}
        \set{\colvec[r]{1 \\ 2},\colvec[r]{2 \\ 4}}
        \qquad
        2\cdot\colvec[r]{1 \\ 2}+0\cdot\colvec[r]{2 \\ 4}
        =0\cdot\colvec[r]{1 \\ 2}+1\cdot\colvec[r]{2 \\ 4}
      \end{equation*}
      Thus, when a subset is not a basis, it can be the case that its
      linear combinations are not unique.

      But just because a subset is not a basis does not imply that its
      combinations must be not unique.
      For instance, this set 
      \begin{equation*}
        \set{\colvec[r]{1 \\ 2}}
      \end{equation*}
      does have the property that
      \begin{equation*}
        c_1\cdot\colvec[r]{1 \\ 2}
        =
        c_2\cdot\colvec[r]{1 \\ 2}
      \end{equation*}
      implies that $c_1=c_2$.
      The idea here is that this subset fails to be a basis because it fails
      to span the space.
\end{Answer}
