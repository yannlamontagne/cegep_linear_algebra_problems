%-----------------------------------------------------
% index key words
%-----------------------------------------------------
\index{basis}
\index{polynomial space}

%-----------------------------------------------------
% name, leave blank
% title, if the exercise has a name i.e. Hilbert's matrix
% difficulty = n, where n is the number of stars
% origin = "\cite{ref}"
%-----------------------------------------------------
\begin{Exercise}[
name={},
title={}, 
difficulty=0,
origin={\cite{JH}}]
Represent the vector with respect to the given basis.
\Question \( \colvec[r]{1 \\ 2} \),
        \( B=\sequence{\colvec[r]{1 \\ 1},\colvec[r]{-1 \\ 1}}\subseteq\Re^2 \)
\Question \( x^2+x^3 \),
        \( D=\sequence{1,\;1+x,\;1+x+x^2,\;1+x+x^2+x^3}\subseteq\polyspace_3 \)
%\Question \( \colvec[r]{0 \\ -1 \\ 0 \\ 1} \),
%       \( \stdbasis_4\subseteq\Re^4 \)
\end{Exercise}

\begin{Answer}
\Question We solve
          \begin{equation*}
            c_1\colvec[r]{1 \\ 1}
            +c_2\colvec[r]{-1 \\ 1}
            =\colvec[r]{1 \\ 2}
          \end{equation*}
          with
          \begin{equation*}
             \begin{amat}[r]{2}
               1  &-1  &1  \\
               1  &1   &2
             \end{amat}
        \end{equation*}
	and Gaussian elimination gives
	\begin{equation*}
             \begin{amat}[r]{2}
               1  &-1  &1  \\
               0  &2   &1
             \end{amat}
          \end{equation*}
          and conclude that \( c_2=1/2 \) and so \( c_1=3/2 \).
          Thus, the representation is this.
          \begin{equation*}
            \rep{\colvec[r]{1 \\ 2}}{B}=\colvec[r]{3/2 \\ 1/2}_B
          \end{equation*}
\Question The relationship
           $c_1\cdot(1)+c_2\cdot(1+x)+c_3\cdot(1+x+x^2)+c_4\cdot(1+x+x^2+x^3)
             =x^2+x^3$
           is easily solved by inspection to give that $c_4=1$, $c_3=0$, $c_2=-1$, and
           $c_1=0$.
           \begin{equation*}
              \rep{x^2+x^3}{D}=\colvec[r]{0 \\ -1 \\ 0 \\ 1}_D
           \end{equation*} 
%        \Question \( \rep{\colvec[r]{0 \\ -1 \\ 0 \\ 1}}{\stdbasis_4}
%                     =\colvec[r]{0 \\ -1 \\ 0 \\ 1}_{\stdbasis_4} \)

\end{Answer}
