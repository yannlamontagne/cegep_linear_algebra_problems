%-----------------------------------------------------
% index key words
%-----------------------------------------------------
\index{basis}
\index{polynomial space}

%-----------------------------------------------------
% name, leave blank
% title, if the exercise has a name i.e. Hilbert's matrix
% difficulty = n, where n is the number of stars
% origin = "\cite{ref}"
%-----------------------------------------------------
\begin{Exercise}[
name={},
title={}, 
difficulty=0,
origin={\cite{JH}}]
Find a basis for each of these subspaces of the space 
$\polyspace_3$ of cubic polynomials. 
\Question The subspace of cubic polynomials $p(x)$ 
such that $p(7)=0$.
\Question The subspace of polynomials $p(x)$ such
that $p(7)=0$ and $p(5)=0$.
\Question The subspace of polynomials $p(x)$ such
that $p(7)=0$, $p(5)=0$, and~$p(3)=0$.
\Question The space of polynomials $p(x)$ such
that $p(7)=0$, $p(5)=0$, $p(3)=0$, and~$p(1)=0$.
\end{Exercise}

\begin{Answer}
        \Question The subspace is this.
          \begin{equation*}
             \set{a_0+a_1x+a_2x^2+a_3x^3 \suchthat a_0+7a_1+49a_2+343a_3=0 }
          \end{equation*}
          Rewriting $a_0=-7a_1-49a_2-343a_3$ gives this.
          \begin{equation*}
             \set{(-7a_1-49a_2-343a_3)+a_1x+a_2x^2+a_3x^3 
               \suchthat a_i\in\Re }
          \end{equation*}
          On breaking out the
          parameters, this suggests \( \sequence{-7+x,-49+x^2,-343+x^3} \)  
          for the basis (it is easily verified).
        \Question The given subspace is the collection of cubics
          $p(x)=a_0+a_1x+a_2x^2+a_3x^3$ such that $a_0+7a_1+49a_2+343a_3=0$
          and $a_0+5a_1+25a_2+125a_3=0$.     
          Gauss's Method 
          \begin{equation*}
            \begin{linsys}{4}
              a_0  &+  &7a_1  &+  &49a_2  &+  &343a_3  &=  &0  \\
              a_0  &+  &5a_1  &+  &25a_2  &+  &125a_3  &=  &0    
            \end{linsys}
        \end{equation*}
	gives
	\begin{equation*}    
	\begin{linsys}{4}
              a_0  &+  &7a_1  &+  &49a_2  &+  &343a_3  &=  &0  \\
                   &   &-2a_1 &-  &24a_2  &-  &218a_3  &=  &0    
            \end{linsys}
          \end{equation*}
          gives that $a_1=-12a_2-109a_3$ and that $a_0=35a_2+420a_3$.
          Rewriting $(35a_2+420a_3)+(-12a_2-109a_3)x+a_2x^2+a_3x^3$
          as $a_2\cdot(35-12x+x^2)+a_3\cdot(420-109x+x^3)$
          suggests this for a basis  $\sequence{35-12x+x^2,420-109x+x^3}$.
          The above shows that it spans the space.
          Checking it is linearly independent is routine.
          (\textit{Comment.} 
          A worthwhile check is to verify that both polynomials in the
          basis have both seven and five as roots.)
        \Question Here there are three conditions on the cubics,
          that $a_0+7a_1+49a_2+343a_3=0$, that $a_0+5a_1+25a_2+125a_3=0$,
          and that $a_0+3a_1+9a_2+27a_3=0$.
          Gauss's Method 
          \begin{equation*}
            \begin{linsys}{4}
              a_0  &+  &7a_1  &+  &49a_2  &+  &343a_3  &=  &0  \\
              a_0  &+  &5a_1  &+  &25a_2  &+  &125a_3  &=  &0  \\    
              a_0  &+  &3a_1  &+  &9a_2   &+  &27a_3   &=  &0    
            \end{linsys}
        \end{equation*}
        gives
        \begin{equation*}       
	\begin{linsys}{4}
              a_0  &+  &7a_1  &+  &49a_2  &+  &343a_3  &=  &0  \\
                   &   &-2a_1 &-  &24a_2  &-  &218a_3  &=  &0  \\    
                   &   &      &   &8a_2   &+  &120a_3  &=  &0    
            \end{linsys}
          \end{equation*}
          yields the single free variable $a_3$, with 
          $a_2=-15a_3$, $a_1=71a_3$, and $a_0=-105a_3$.
          The parametrization is this.
          \begin{multline*} 
            \set{(-105a_3)+(71a_3)x+(-15a_3)x^2+(a_3)x^3\suchthat a_3\in\Re} \\
            =
            \set{a_3\cdot(-105+71x-15x^2+x^3)\suchthat a_3\in\Re}
          \end{multline*}
          Therefore, a natural candidate for the basis is 
          $\sequence{-105+71x-15x^2+x^3}$.
          It spans the space by the work above.
          It is clearly linearly independent because it is a one-element
          set (with that single element not the zero object of the space).
          Thus, any cubic through the three points $(7,0)$, $(5,0)$, and
          $(3,0)$ is a multiple of this one.
          (\textit{Comment.}
          As in the prior question, 
          a worthwhile check is to verify that plugging seven, five, and
          three into this polynomial yields zero each time.)
\Question This is the trivial subspace of $\polyspace_3$.
          Thus, the  basis is empty $\sequence{}$.

\end{Answer}
