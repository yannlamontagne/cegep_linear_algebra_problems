%-----------------------------------------------------
% index key words
%-----------------------------------------------------
\index{basis}

%-----------------------------------------------------
% name, leave blank
% title, if the exercise has a name i.e. Hilbert's matrix
% difficulty = n, where n is the number of stars
% origin = "\cite{ref}"
%-----------------------------------------------------
\begin{Exercise}[
name={},
title={}, 
difficulty=0,
origin={\cite{JH}}]
Determine if each is a basis for \( \Re^3 \).
\begin{multicols}{2}
\Question \( \sequence{
                 \colvec[r]{1 \\ 2 \\ 3},
                 \colvec[r]{3 \\ 2 \\ 1},
                 \colvec[r]{0 \\ 0 \\ 1}}  \)
\Question \( \sequence{
                 \colvec[r]{1 \\ 2 \\ 3},
                 \colvec[r]{3 \\ 2 \\ 1}}  \)
\Question \( \sequence{
                 \colvec[r]{0 \\ 2 \\ -1},
                 \colvec[r]{1 \\ 1 \\ 1},
                 \colvec[r]{2 \\ 5 \\ 0}}  \)
\Question \( \sequence{
                 \colvec[r]{0 \\ 2 \\ -1},
                 \colvec[r]{1 \\ 1 \\ 1},
                 \colvec[r]{1 \\ 3 \\ 0}}  \)
\EndCurrentQuestion
\end{multicols}

\end{Exercise}

\begin{Answer}
      Each set is a basis if and only
      if we can express each vector in the space in a unique way as a linear
      combination of the given vectors.
        \Question Yes this is a basis.
          The relation
          \begin{equation*}
            c_1\colvec[r]{1 \\ 2 \\ 3}
            +c_2\colvec[r]{3 \\ 2 \\ 1}
            +c_3\colvec[r]{0 \\ 0 \\ 1}
            =\colvec{x \\ y \\ z}
          \end{equation*}
          gives
          \begin{equation*}
            \begin{amat}{3}
              1  &3  &0  &x  \\
              2  &2  &0  &y  \\
              3  &1  &1  &z
            \end{amat}
                \end{equation*}
        and Gaussian elimination gives 
	\begin{equation*}    
	\begin{amat}{3}
              1  &3  &0  &x \hfill\hbox{}  \\
              0  &-4 &0  &-2x+y \hfill\hbox{}  \\
              0  &0  &1  &x-2y+z 
            \end{amat}
          \end{equation*}
          which has the unique solution
          \( c_3=x-2y+z \), \( c_2=x/2-y/4 \), and
          \( c_1=-x/2+3y/4 \).
        \Question This is not a basis.
          Setting it up as in the prior part
          \begin{equation*}
            c_1\colvec[r]{1 \\ 2 \\ 3}
            +c_2\colvec[r]{3 \\ 2 \\ 1}
            =\colvec{x \\ y \\ z}
          \end{equation*}
          gives a linear system whose solution
          \begin{equation*}
            \begin{amat}{2}
              1  &3  &x  \\ 
              2  &2  &y  \\
              3  &1  &z
            \end{amat}
        \end{equation*}
        gives
         \begin{equation*}    
	\begin{amat}{2}
              1  &3  &x\hfill\hbox{}  \\ 
              0  &-4 &-2x+y\hfill\hbox{}  \\
              0  &0  &x-2y+z
            \end{amat}
          \end{equation*}
          is possible if and only if the three-tall vector's components
          $x$, $y$, and $z$ satisfy $x-2y+z=0$.
          For instance, we can find the coefficients $c_1$ and $c_2$ that
          work when $x=1$, $y=1$, and $z=1$.
          However, there are no $c$'s that work for
          $x=1$, $y=1$, and $z=2$.
          Thus this is not a basis; it does not span the space.
        \Question Yes, this is a basis.
         Setting up the relationship leads to this reduction
          \begin{equation*}
            \begin{amat}{3}
              0  &1  &2  &x \hfill\hbox{} \\
              2  &1  &5  &y \hfill\hbox{} \\
             -1  &1  &0  &z 
            \end{amat}
        \end{equation*}
        gives
         \begin{equation*}            
\begin{amat}{3}
             -1  &1  &0   &z  \hfill\hbox{} \\
              0  &3  &5   &y+2z \hfill\hbox{} \\
              0  &0  &1/3 &x-y/3-2z/3
            \end{amat}
          \end{equation*}
          which has a unique solution for each triple of components
          $x$, $y$, and $z$.
        \Question No, this is not a basis.
          The reduction of
          \begin{equation*}
            \begin{amat}{3}
              0  &1  &1  &x   \\
              2  &1  &3  &y   \\
             -1  &1  &0  &z  
            \end{amat}
        \end{equation*}
	gives
	 \begin{equation*}    
	\begin{amat}{3}
             -1  &1  &0  &z  \hfill\hbox{} \\
              0  &3  &3  &y+2z  \hfill\hbox{} \\
              0  &0  &0  &x-y/3-2z/3
            \end{amat}
          \end{equation*}
          which does not have a solution for each triple $x$, $y$, and $z$.
          Instead, the span of the given set 
          includes only those vectors where \( x=y/3+2z/3 \).

\end{Answer}
