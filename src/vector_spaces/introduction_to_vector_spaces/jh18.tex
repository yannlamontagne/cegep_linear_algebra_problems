%-----------------------------------------------------
% index key words
%-----------------------------------------------------
\index{vector space}

%-----------------------------------------------------
% name, leave blank
% title, if the exercise has a name i.e. Hilbert's matrix
% difficulty = n, where n is the number of stars
% origin = "\cite{ref}"
%-----------------------------------------------------
\begin{Exercise}[
name={},
title={}, 
difficulty=0,
origin={\cite{JH}}]

Assume that \( \vec{v}\in V \) is not \( \zero \).

\Question Prove that \( r\cdot\vec{v}=\zero \) if and only if \( r=0 \).
\Question Prove that \( r_1\cdot\vec{v}=r_2\cdot\vec{v} \) if
and only if \( r_1=r_2 \).
\Question Prove that any nontrivial vector space is infinite.

 
\end{Exercise}

\begin{Answer}

Assume that \( \vec{v}\in V \) is not \( \zero \).
        
\Question One direction of the if and only if is clear:~if $r=0$
then $r\cdot\vec{v}=\zero$.
For the other way, let \( r \) be a nonzero scalar.
If \( r\vec{v}=\zero \) then
\( (1/r)\cdot r\vec{v}=(1/r)\cdot \zero \) shows that
$\vec{v}=\zero$,  contrary to the assumption.

\Question Where \( r_1,r_2 \) are scalars,
\( r_1\vec{v}=r_2\vec{v}\, \)
holds if and only if \( (r_1-r_2)\vec{v}=\zero \).
By the prior item, then \( r_1-r_2=0 \).

\Question A nontrivial space has a vector
\( \vec{v}\neq\zero \).
Consider the set \( \set{k\cdot\vec{v}\suchthat k\in\Re} \).
By the prior item this set is infinite.

\end{Answer}
