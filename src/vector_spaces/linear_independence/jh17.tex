%-----------------------------------------------------
% index key words
%-----------------------------------------------------
\index{linear independence}
\index{linear dependence}

%-----------------------------------------------------
% name, leave blank
% title, if the exercise has a name i.e. Hilbert's matrix
% difficulty = n, where n is the number of stars
% origin = "\cite{ref}"
%-----------------------------------------------------
\begin{Exercise}[
name={},
title={}, 
difficulty=0,
origin={\cite{JH}}]
\Question We might conjecture that the union $S\union T$ of 
          linearly independent sets 
          is linearly independent if and only if their spans have a trivial 
          intersection $\spanof{S}\intersection \spanof{T}=\set{\zero}$.
          What is wrong with this argument for the `if' direction
          of that conjecture? 
          ``If the union $S\union T$ is linearly independent then
          the only solution to          
          $c_1\vec{s}_1+\cdots+c_n\vec{s}_n
            +d_1\vec{t}_1+\cdots+d_m\vec{t}_m
            =\zero$
          is the trivial one $c_1=0$, \ldots, $d_m=0$. 
          So any member of the intersection of the spans
          must be the zero vector because in 
          $c_1\vec{s}_1+\cdots+c_n\vec{s}_n
            =d_1\vec{t}_1+\cdots+d_m\vec{t}_m$
          each scalar is zero.''
\Question Give an example showing that the conjecture is false.
\Question Find linearly independent sets \( S \) and \( T \)
          so that the union of \( S-(S\cap T)\) and \( T-(S\cap T) \)
          is linearly independent, but the union \( S\cup T \) is
          not linearly independent.
\Question Characterize when the union of two linearly independent sets
          is linearly independent, in terms of the intersection of spans.
\end{Exercise}

\begin{Answer}
\Question The vectors 
          $\vec{s}_1,\ldots,\vec{s}_n,\vec{t}_1,\ldots,\vec{t}_m$
          are distinct. 
          But we could have that the union $S\cup T$ is linearly 
          independent with some $\vec{s}_i$ equal to some $\vec{t}_j$.
\Question  One example in \( \Re^2 \) is these two.
          \begin{equation*}
            S=\set{\colvec{1 \\ 0}}
            \qquad
            T=\set{\colvec{1 \\ 0}, \colvec{0 \\ 1}}
          \end{equation*}
\Question 
          An example from \( \Re^2 \) is these sets.
          \begin{equation*}
            S=\set{\colvec{1 \\ 0}, \colvec{0 \\ 1}}
            \qquad
            T=\set{\colvec{1 \\ 0}, \colvec{1 \\ 1}}
          \end{equation*}
\Question The union of two linearly independent sets $S\union T$
          is linearly independent if and only if their spans of \( S \) 
          and \( T-(S\cap T)\) have a trivial intersection 
          $\spanof{S}\intersection \spanof{T-(S\cap T)}=\set{\zero}$.
          To prove that, assume that \( S \) and \( T \) are linearly 
          independent subsets of some vector space.

          For the `only if' direction, assume that the intersection of
          the spans is trivial 
          \( \spanof{S}\intersection \spanof{T-(S\cap T)}=\set{\zero} \).
          Consider the set $S\union (T-(S\cap T))=S\union T$ and
          consider the linear relationship 
          $c_1\vec{s}_1+\dots+c_n\vec{s}_n
            +d_1\vec{t}_1+\dots+d_m\vec{t}_m=\zero$.
          Subtracting gives
          $c_1\vec{s}_1+\dots+c_n\vec{s}_n=
            -d_1\vec{t}_1-\dots-d_m\vec{t}_m$.
          The left side of that equation sums to a vector in $\spanof{S}$, and
          the right side is a vector in $\spanof{T-(S\cap T)}$.
          Therefore, since the intersection of the spans is trivial, both
          sides equal the zero vector.
          Because $S$ is linearly independent, all of the $c$'s are zero.
          Because $T$ is linearly independent so also is  
          $T-(S\cap T)$ linearly independent, 
          and therefore all of the $d$'s are zero.
          Thus, the original linear relationship among members of 
          $S\union T$ only holds if all of the coefficients are zero.
          Hence, $S\union T$ is linearly independent.

          For the `if' half we can make the same argument in reverse.
          Suppose that the union $S\union T$ is linearly independent.
          Consider a linear relationship among members of 
          $S$ and $T-(S\cap T)$.        
          $c_1\vec{s}_1+\cdots+c_n\vec{s}_n
            +d_1\vec{t}_1+\cdots+d_m\vec{t}_m
            =\zero$
          Note that no $\vec{s}_i$ is equal to a $\vec{t}_j$
          so that is a combination of distinct vectors. 
          So the only solution 
          is the trivial one $c_1=0$, \ldots, $d_m=0$. 
          Since any vector $\vec{v}$ 
          in the intersection of the spans
          $\spanof{S}\intersection \spanof{T-(S\cap T)}$ 
          we can write 
          $\vec{v}=c_1\vec{s}_1+\cdots+c_n\vec{s}_n
            =-d_1\vec{t}_1-\cdots-d_m\vec{t}_m$,
          and it must be the zero vector because each scalar is zero.
\end{Answer}
