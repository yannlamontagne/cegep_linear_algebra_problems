%-----------------------------------------------------
% index key words
%-----------------------------------------------------
\index{linear independence}
\index{linear dependence}

%-----------------------------------------------------
% name, leave blank
% title, if the exercise has a name i.e. Hilbert's matrix
% difficulty = n, where n is the number of stars
% origin = "\cite{ref}"
%-----------------------------------------------------
\begin{Exercise}[
name={},
title={}, 
difficulty=0,
origin={\cite{JH}}]
Determine whether each subset of \( \Re^3 \) is linearly dependent or
    linearly independent. 
\Question \( \set{\colvec{1 \\ -3 \\ 5},
                    \colvec{2 \\ 2 \\ 4},
                    \colvec{4 \\ -4 \\ 14} }  \)
\Question \( \set{\colvec{1 \\ 7 \\ 7},
                    \colvec{2 \\ 7 \\ 7},
                    \colvec{3 \\ 7 \\ 7} }  \)
\Question \( \set{\colvec{0 \\ 0 \\ -1},
                    \colvec{1 \\ 0 \\ 4} }  \)
\Question \( \set{\colvec{9 \\ 9 \\ 0},
                    \colvec{2 \\ 0 \\ 1},
                    \colvec{3 \\ 5 \\ -4},
                    \colvec{12 \\ 12 \\ -1} }  \)
\end{Exercise}

\begin{Answer}
\Question It is dependent.
          Considering
          \begin{equation*}
             c_1\colvec{1 \\ -3 \\ 5}
             +c_2\colvec{2 \\ 2 \\ 4}
             +c_3\colvec{4 \\ -4 \\ 14}
             =\colvec{0 \\ 0 \\ 0}
          \end{equation*}
          gives this linear system.
          \begin{equation*}
            \begin{linsys}{3}
              c_1  &+  &2c_2  &+  &4c_3  &=  &0  \\
              -3c_1&+  &2c_2  &-  &4c_3  &=  &0  \\
              5c_1 &+  &4c_2  &+  &14c_3 &=  &0  
            \end{linsys}
          \end{equation*}
          Gauss's Method 
          \begin{equation*}
            \begin{amat}[r]{3}
              1  &2  &4  &0  \\
              -3 &2  &-4 &0  \\
              5  &4  &14 &0
            \end{amat}
	\end{equation*}
	gives
        \begin{equation*}    
	\begin{amat}[r]{3}
              1  &2  &4  &0  \\
              0  &8  &8  &0  \\
              0  &0  &0  &0
            \end{amat}
          \end{equation*}
          yields a free variable, so there are infinitely many solutions.
          For an example of a particular dependence we can set $c_3$ to be,
          say, $1$.  Then we get
          \( c_2=-1 \) and \( c_1=-2 \).
        \Question It is dependent.
          The linear system that arises here
          \begin{equation*}
            \begin{amat}[r]{3}
              1  &2  &3  &0  \\
              7  &7  &7  &0  \\
              7  &7  &7  &0
            \end{amat}
        \end{equation*}
        and Gaussian elimination gives
        \begin{equation*}    
	\begin{amat}[r]{3}
              1  &2  &3   &0  \\
              0  &-7 &-14 &0  \\
              0  &0  &0   &0
            \end{amat}
          \end{equation*}
          has infinitely many solutions.
          We can get a particular solution by taking $c_3$ to be, say,
          $1$, and back-substituting to get the resulting $c_2$ and $c_1$.
        \Question It is linearly independent.
          The system
          \begin{equation*}
            \begin{amat}[r]{2}
              0  &1  &0  \\
              0  &0  &0  \\
              -1 &4  &0
            \end{amat}
        \end{equation*}
        and Gaussian elimination gives
        \begin{equation*}     
\begin{amat}{2}
              -1 &4  &0  \\
              0  &1  &0  \\
              0  &0  &0  
            \end{amat}
          \end{equation*}
          has only the solution $c_1=0$ and $c_2=0$.
          (We could also have gotten the answer by inspection the second
          vector is obviously not a multiple of the first, and vice versa.)
        \Question It is linearly dependent.
          The linear system
          \begin{equation*}
            \begin{amat}[r]{4}
              9  &2  &3  &12  &0  \\
              9  &0  &5  &12  &0  \\
              0  &1  &-4 &-1  &0
            \end{amat}
          \end{equation*}
          has more unknowns than equations, and so Gauss's Method
          must end with at least one variable free (there can't be a 
          contradictory equation because the system is homogeneous, and so
          has at least the solution of all zeroes).
          To exhibit a combination, we can do the reduction 
          \begin{equation*}
            \begin{amat}[r]{4}
              9  &2  &3  &12  &0  \\
              0  &-2 &2  &0   &0  \\
              0  &0  &-3 &-1  &0
            \end{amat}
          \end{equation*}
          and take, say,  $c_4=1$.
          Then we have that $c_3=-1/3$, $c_2=-1/3$, and $c_1=-31/27$.

\end{Answer}
