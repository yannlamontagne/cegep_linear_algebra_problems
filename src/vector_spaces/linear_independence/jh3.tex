%-----------------------------------------------------
% index key words
%-----------------------------------------------------
\index{linear independence}
\index{linear dependence}
\index{function space}

%-----------------------------------------------------
% name, leave blank
% title, if the exercise has a name i.e. Hilbert's matrix
% difficulty = n, where n is the number of stars
% origin = "\cite{ref}"
%-----------------------------------------------------
\begin{Exercise}[
name={},
title={}, 
difficulty=0,
origin={\cite{JH}}]
Prove that each set \( \set{f,\;g} \) is linearly independent in the
    vector space of all functions from \( \Re^+ \) to \( \Re \).
\Question \( f(x)=x \) and \( g(x)=1/x \)
\Question \( f(x)=\cos(x) \) and \( g(x)=\sin(x) \)
\Question \( f(x)=e^x \) and \( g(x)=\ln(x) \)
\end{Exercise}

\begin{Answer}
      Let $Z$ be the zero function $Z(x)=0$, which is the additive identity in
      the vector space under discussion.
        \Question This set is linearly independent.  
          Consider \( c_1\cdot f(x)+c_2\cdot g(x)=Z(x) \).
          Plugging in \( x=1 \) and \( x=2 \) gives a linear system 
          \begin{equation*}
            \begin{linsys}{2}
              c_1\cdot 1  &+  &c_2\cdot 1     &=  &0  \\
              c_1\cdot 2  &+  &c_2\cdot (1/2) &=  &0
            \end{linsys}
          \end{equation*}
          with the unique solution \( c_1=0 \), \( c_2=0 \).
        \Question This set is linearly independent.  
          Consider \( c_1\cdot f(x)+c_2\cdot g(x)=Z(x) \) and 
          plug in \( x=0 \) and \( x=\pi/2 \) to get 
          \begin{equation*}
            \begin{linsys}{2}
              c_1\cdot 1  &+  &c_2\cdot 0     &=  &0  \\
              c_1\cdot 0  &+  &c_2\cdot 1     &=  &0
            \end{linsys}
          \end{equation*}
          which obviously gives that \( c_1=0 \), \( c_2=0 \).
        \Question This set is also linearly independent.  
          Considering \( c_1\cdot f(x)+c_2\cdot g(x)=Z(x) \) and 
          plugging in \( x=1 \) and \( x=e \) 
          \begin{equation*}
            \begin{linsys}{2}
              c_1\cdot e    &+  &c_2\cdot 0     &=  &0  \\
              c_1\cdot e^e  &+  &c_2\cdot 1     &=  &0
            \end{linsys}
          \end{equation*}
          gives that \( c_1=0 \) and \( c_2=0 \).

\end{Answer}
