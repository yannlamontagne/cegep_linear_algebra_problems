%-----------------------------------------------------
% index key words
%-----------------------------------------------------
\index{dimension}

%-----------------------------------------------------
% name, leave blank
% title, if the exercise has a name i.e. Hilbert's matrix
% difficulty = n, where n is the number of stars
% origin = "\cite{ref}"
%-----------------------------------------------------
\begin{Exercise}[
name={},
title={}, 
difficulty=0,
origin={\cite{JH}}]
    Prove that if \( U \) and \( W \) are both three-dimensional
    subspaces of \( \Re^5 \) then \( U\intersection W \) is non-trivial.
    State a generalization of the above.
\end{Exercise}

\begin{Answer}
       Let \( B_U \) be a basis for \( U \) and let \( B_W \)
       be a basis for \( W \).
       Consider the concatenation of the two basis sequences.
       If there is a repeated element then the intersection
       \( U\intersection W \) is nontrivial.
       Otherwise, 
       the set \( B_U\union B_W \) is linearly dependent as it is a
       six member subset of the five-dimensional space \( \Re^5 \).
       In either case some member of \( B_W \) is in the span of \( B_U \), and
       thus \( U\intersection W \) is more than just the trivial space
       \( \set{\zero\,} \).

       Generalization:
       if \( U,W \) are subspaces of a vector space of dimension \( n \) and
       if \( \dim(U)+\dim(W)>n \) then they have a nontrivial
       intersection.  
\end{Answer}
