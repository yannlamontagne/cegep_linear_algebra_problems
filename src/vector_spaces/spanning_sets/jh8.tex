%-----------------------------------------------------
% index key words
%-----------------------------------------------------
\index{subspace}
\index{spaning set}

%-----------------------------------------------------
% name, leave blank
% title, if the exercise has a name i.e. Hilbert's matrix
% difficulty = n, where n is the number of stars
% origin = "\cite{ref}"
%-----------------------------------------------------
\begin{Exercise}[
name={},
title={}, 
difficulty=0,
origin={\cite{JH}}]
Does the span of a set depend on the enclosing space?
    That is, if \( W \) is a subspace of \( V \) and \( S \) is a subset of
    \( W \) (and so also a subset of \( V \)), might the span of \( S \) in
    \( W \) differ from the span of \( S \) in \( V \)?
\end{Exercise}

\begin{Answer}
      The span of a set does not depend on the enclosing space.
      A linear combination of vectors from \( S \) gives the same sum
      whether we regard the operations as those of \( W \) or as those of
      \( V \), because the operations of \( W \) are inherited from \(  V \).  


\end{Answer}
